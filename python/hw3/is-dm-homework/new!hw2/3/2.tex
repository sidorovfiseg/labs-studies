\question
Однажды знаменитому Пэркюлю Уаро довелось расследовать удивительное дело о пропаже буквы $B$ и страшной смерти буквы $A$. Проведя весь день за допросами, Пэркюль смог определить главного подозреваемого, однако для окончательного вердикта пришлось записать в дневник следующее:
\\
\\
«Ни в коем случае не мог лежать на трубе нож, и в то же время труба быть мокрой, как и наоборот. Также обязательно должен был быть путь отступления, либо сообщник. Только тогда буква Б – преступник»
\\
\\
Помогите детективу, записав высказывание «Буква Б – преступник» в виде булевой функции, где $A$ – труба была мокрой, $B$ – на трубе лежал нож, $C$ – был путь отступления, $D$ – был сообщник. А также:
\begin{enumerate}
    \item Составьте СДНФ и СКНФ получившейся функции.
    \item Составьте полином Жегалкина получившейся функции любыми 2 способами
\end{enumerate}

---------------

Автор -- Константин Васильев, М3213