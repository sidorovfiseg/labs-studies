\question
В университете MIT(O) для удобства работы с цифровыми данными у каждого студента есть свой уникальный идентификационный номер ИСУ: \{1, 2, 3… 9\}. В вузе есть различные клубы из студентов:
\begin{itemize}
    \item клуб любителей мат. анализа (обознач. буквой $M$), состоит из студентов: \{1, 2, 5, 6\}
    \item клуб любителей лин. алгебры (обознач. буквой $L$), состоит из студентов: \{1, 7, 5, 9, 4\}
    \item клуб любителей алгоритмов (обознач. буквой $A$), состоит из студентов: \{2, 8, 3, 1\}
    \item клуб любителей программирования (обознач. буквой $P$), состоит из студентов: \{1, 2, 6, 3, 8, 9\}
\end{itemize}
Студент Вася очень любит ДМ, и поэтому он захотел создать клуб любителей дискретной математики. Для создания клуба необходимо отправить письмо в студ. офис, указав там список участников. Но Вася решил продемонстрировать свои знания дискретки, и отправил вместо списка эту записку:
\begin{center}
"множество участников клуба -- это $X$, где $M \cap A \cup M \cap P \cap A \cup (L \cap (\overline{L} \cup P))$"
\end{center}
\\
Помогите студ. офису составить список участников клуба, упростив выражение Васи.

---------------

Автор -- Тимур Гонтарь, М3206