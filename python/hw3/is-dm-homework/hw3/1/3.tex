\question
Нарисуйте  по степеням вершин простой неориентированный граф (исходный граф:  6 4 3 3 4 4 2 ), желательно состоящий из одной компоненты связности. 
К полученному графу постройте следующие графы (обязательно доказывайте и  обосновывайте верность своих построений в решении):
\begin{enumerate}
\item   частичный граф к исходному, который будет являться двудольным
\item   подграф для построенного, такой чтобы при пересечении полученного подграфа и ранее построенного частичного графа - получалось дерево;
\item   надграф к исходному, такой чтобы  дополнительный граф к надграфу был  регулярным;
\item   2 изоморфных графа с исходным, причем один с планарной укладкой, а другой с непланарной;
\item   дополнительные графы к полученным выше частичному и подграфу относительно исходного графа;
\item   нуль - граф к полученному выше надграфу;
\end{enumerate}
дополнительно: отметьте на полученных графах точки сочленения и мосты.





